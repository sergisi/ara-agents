\documentclass{tufte-handout}
%\usepackage[ruled,vlined,slide]{algorithm2e}
\usepackage[utf8]{inputenc}
\usepackage{color}

\usepackage[english]{babel}

\usepackage{listings}
\lstset{language=Python}
%\lstset{basicstyle=\tt\footnotesize}
\lstset{basicstyle=\tt\small}
\lstset{tabsize=2}
\lstset{commentstyle=\textit}
\lstset{extendedchars=true}

\lstset{numbers=left,       % where to put the line-numbers
numberstyle=\footnotesize,  % the size of the fonts that are used for the line-numbers
stepnumber=1,             % the step between two line-numbers. If it's 1, each line
                          % will be numbered
numbersep=5pt,                  % how far the line-numbers are from the code
backgroundcolor=\color{lightgray},  % choose the background color. You must add \usepackage{color}
showspaces=false,               % show spaces adding particular underscores
showstringspaces=false,         % underline spaces within strings
showtabs=false,           % show tabs within strings adding particular under scores
frame=single,                   % adds a frame around the code
captionpos=b,                   % sets the caption-position to bottom
breaklines=true,                % sets automatic line breaking
breakatwhitespace=false,      % sets if automatic breaks should only happen at whitespace
title=\lstname,        % show the filename of files included with \lstinputlisting;
                       % also try caption instead of title
}


%\geometry{showframe}% for debugging purposes -- displays the margins

\usepackage{amsmath}

% Set up the images/graphics package
\usepackage{graphicx}
\setkeys{Gin}{width=\linewidth,totalheight=\textheight,keepaspectratio}
\graphicspath{{graphics/}}

\title[HW 1: Knowledge Based Agents]{Automatic Reasoning and Learning \\-
Homework Assignment 1: Knowledge Based Agents with  CP0 Logic}
\author[ramon]{Ram\'on B\'ejar}
% \date{28 March 2019}  % if the \date{} command is left out, the current date will  be used

% The following package makes prettier tables.  We're all about the bling!
\usepackage{booktabs}

% The units package provides nice, non-stacked fractions and better spacing
% for units.
\usepackage{units}

% The fancyvrb package lets us customize the formatting of verbatim
% environments.  We use a slightly smaller font.
\usepackage{fancyvrb}
\fvset{fontsize=\normalsize}

% Small sections of multiple columns
\usepackage{multicol}

% Provides paragraphs of dummy text
\usepackage{lipsum}

% These commands are used to pretty-print LaTeX commands
\newcommand{\doccmd}[1]{\texttt{\textbackslash#1}}% command name -- adds backslash automatically
\newcommand{\docopt}[1]{\ensuremath{\langle}\textrm{\textit{#1}}\ensuremath{\rangl
e}}% optional command argument
\newcommand{\docarg}[1]{\textrm{\textit{#1}}}% (required) command argument
\newenvironment{docspec}{\begin{quote}\noindent}{\end{quote}}% command specification environment
\newcommand{\docenv}[1]{\textsf{#1}}% environment name
\newcommand{\docpkg}[1]{\texttt{#1}}% package name
\newcommand{\doccls}[1]{\texttt{#1}}% document class name
\newcommand{\docclsopt}[1]{\texttt{#1}}% document class option name


\usepackage{lipsum, ifluatex, ifxetex}
%Next block avoids bug, from  http://tex.stackexchange.com/a/200725/1913
% \renewcommand\allcapsspacing[1]{{\addfontfeature{LetterSpace=15}#1}}
 % \renewcommand\smallcapsspacing[1]{{\addfontfeature{LetterSpace=10}#1}}

%\ifxetex
%  \newcommand{\textls}[2][5]{%
%    \begingroup\addfontfeatures{LetterSpace=#1}#2\endgroup
%  }
%  \renewcommand{\allcapsspacing}[1]{\textls[15]{#1}}
%  \renewcommand{\smallcapsspacing}[1]{\textls[10]{#1}}
%  \renewcommand{\allcaps}[1]{\textls[15]{\MakeTextUppercase{#1}}}
%  \renewcommand{\smallcaps}[1]{\smallcapsspacing{\scshape\MakeTextLowercase{#1}}}
%  \renewcommand{\textsc}[1]{\smallcapsspacing{\textsmallcaps{#1}}}
% \fi



\begin{document}
\maketitle

\section{Goal}

\begin{fullwidth}
The goal of this work is to develop an intelligent agent for
the Treasure World problem we have presented at one on-line class.

You have to develop your program using the java classes I have provided
with this assignment, where some functions must be implemented, the
finished ones can be modified, and new ones added.
\textcolor{red}{BEWARE:} Everything that I ask in this document that appears in
\textcolor{red}{red} is something that is mandatory to satisfy in order to have
your project evaluated. So, any not satisfied  red point will make the grade of this
project equal to 0. So, before delivering your project, check that you satisfy all these
minimal points. Check it with me, before delivering it, if you want to be sure.
% \end{fullwidth}

Your application works with two main objects: the treasure finder and the world environment. It has to work with the following input:
\begin{enumerate}
\item The dimension of the world (the value of $n$ for the
$n\times n$ Treasure World) (this information is needed by both
the finder agent and the world environment object).
\item This information is used only for the finder agent:
\begin{enumerate}
\item Number of steps $(l)$ to perform.
\item A sequence of $l$ steps of this form:
$$
 x_1,y_1 \ x_2,y_2 \ ... \ x_l,y_l
$$
where $ x_t,y_t $ indicates that at time step $t$ the agent moves
 to position $(x,y)$.
This sequence of steps will be stored in an text file, in a single line.
\end{enumerate}
\item This information is used only for the  world environment object:
\begin{enumerate}
\item The position $x,y$ where the treasure is located.
\item The list of pirate locations, following the same format that the sequence
of steps: in a file with a single line with the positions as:
$$
 px_1,py_1 \ px_2,py_2 \ ... \ px_l,py_l
$$
\end{enumerate}
\end{enumerate}


%~\footnote{You can assume that any input sequence for your program will be a
%{\bf valid and consistent sequence of readings}, so it will never happen that
%from a sequence of readings you can derive contradictory conclusions for a
%same location. So, when you test your program use {\bf only} input sequences
%with this property}.

% \end{enumerate}

% \begin{fullwidth}
With that input, your finder agent should print at the standard output
 (the screen)
the knowledge state for possible locations of the treasure that the agent
has {\bf after it processes each step of the agent}.
This knowledge state will be presented as the $n \times n$
matrix with the $?$ and $X$ symbols, where $?$ indicates a possible location
and $X$ a not possible location. For using this representation of knowledge states, your
agent will use the class {\tt TFState}, that is already implemented, but it can
modified if you need to.
\end{fullwidth}


%%%%%%%%%%%%%%%%%%%%%%%%%%%%%%%%%%%%%%%%%%%%%%%%%%%%%%%%%%
\section{Requirements}

\begin{fullwidth}
You have to satisfy the following requeriments:
\begin{itemize}
\item \textcolor{red}{Use a clean TOP-DOWN design, with small member functions in your
agent classes, such that each function performs a well defined function.
You must comment each member function, explaining what the function
does, its input arguments and its output (if any).
Comment all the function headers and relevant class variables using javadoc
comments.
All your code must contain enough comments so that you can convince
me that you really understand how your program works.}
% A bad design that makes difficult to understand how your agents work
% may produce the grade to be as low as a 6. }
\item \textcolor{red}{Present all your code well organized, using a consistent style of indentation. Use clear and informative names for the variables you use in your
class and class functions}.
\item Your finder agent must use {\bf propositional logic} to reason
 about the possible locations of the treasure. So, the architecture
of your agent will follow the one we have seen at the classroom
 (check the slides about knowledge based systems with propositional
logic) for intelligent agents based on propositional
logic, so the main process that your finder agent must   implement is:
\begin{enumerate}
\item When no inputs have been processed, the only knowledge of
the agent is the original formula $\Gamma$ you have created for
the $n\times n$ world.
\item Each time $t$ the agent receives new information (detector sensor information
or answer from a pirate),  your agent must:
\begin{enumerate}
\item For any location $(x',y')$ of the world, ask whether
it is not possible that the treasure is in that location, that is,
the agent checks if:
$$ \Gamma \cup E \models \neg t_{x',y'}^{t+1}  $$
holds for its current knowledge formula $\Gamma$, where $E$ represents
the information the agent has obtained from the information of the sensor,
 but expressed in propositional logic (the evidence about the world).
As you know, you can perform the inference questions using a
 SAT solver. Use the {\tt sat4j} library (mainly using the ISolver interface at org.sat4j.specs), or any other external SAT solver that you want.
\item Update the knowledge $\Gamma$ of the agent that
is true so far  incorporating
all the clauses corresponding to the positions that have been
 inferred as {\bf not possible locations}. That is, add all the
clauses of the set:
$$
 \{ \ (\neg t_{x',y'}^{t-1}) \ | \ \Gamma \cup E
    \models \neg  t_{x',y'}^{t+1} \} $$
So at the end of the iteration the knowledge formula $\Gamma$ is
updated with new information (or just before performing the next one).
Observe that any location $(x',y')$ that was previously not possible for the treasure
(so $ \neg t_{x',y'}^{t-1} $ was already a clause in $\Gamma$ at the beginning of
the iteration), will be also not possible at time step $t+1$.
\end{enumerate}
\end{enumerate}
\item  \textcolor{red}{You must implement a minimal set of testing functions in the class
{\tt TreasureFinderTest.java}, for testing all the example step sequences I will provide,
 using {\tt junit4}}. This class has some functions
implemented, but some must be finished and you can add any other functions you need for this
testing class.
%Additionally, to get a better grade for this work, you can also provide unit testing functions
%in the class to test more basic functionalities of the class, like for example the functions
%used to check logical consequences of the logical formula of the agent ($\Gamma$).
%If you only provide testing at the level of checking step sequences with their target states
% (following the example provided), the maximum grade of this project will be 9 points.
\end{itemize}
\end{fullwidth}

\section{Minimal set of functions}

\begin{fullwidth}
This is the minimal mandatory set of functions that you must implement in the class TreasureFinder
(check the javadoc comments at the headers of such functions for explanations):
\begin{itemize}
% \item {\tt public AMessage DetectsAt( )}.
\item {\tt public void processDetectorSensorAnswer( AMessage ans )}.
\item {\tt public void processPirateAnswer( AMessage ans )}.
% \item {\tt public void addDetectorSensorEvidence(  )}.
\item {\tt  public void addLastFutureClausesToPastClauses() }.
\item {\tt public void  performInferenceQuestions()}.
% \item {\tt public void sounds\_implications( )}.
\end{itemize}
In the class {\tt TreasureWorldEnv} you must implement:
\begin{itemize}
\item {\tt acceptMessage(AMessage msg)}. Because the provided implementation only works with
the {\tt moveto} message. You must extend it to accept and answer to the other messages:
{\tt detectsat} and {\tt treasureup}.
\item {\tt loadPiratesLocations( String piratesFile )}.
\end{itemize}
In the class {\tt TreasureWorld} (main class of the program) you must implement:
\begin{itemize}
\item {\tt runStepsSequence( int wDim, int tX, int tY,
                              int numSteps, String fileSteps, String piratesFile )}.
\item {\tt void main ( String[] args)}.
\end{itemize}
And this for the test class TreasureFinderTest:
\begin{itemize}
\item {\tt public void testMakeSimpleStep(  TreasureFinder tAgent,
                                          TFState targetState )}.
\item {\tt  testMakeSeqOfSteps( int wDim, int tX, int tY,
                                 int numSteps, String fileSteps, String piratesFile )}.
\end{itemize}

%\begin{fullwidth}
In the test class, there is an example test (function {\tt  TWorldTest1()}), that uses
{\tt testMakeSeqOfSteps} to implement one step sequence test. You can use this example to build
the other tests I will ask, or use some kind of parameterized tests to implement all of them.
All the other existing functions can be modified to fit the needs of your design,
and you can add any  additional functions you need.
\end{fullwidth}


\section{What you Have to Deliver}

\begin{fullwidth}
You must deliver:
\begin{itemize}
\item  \textcolor{red}{{\bf maven build file}. I have to be able to build your
 application with the maven file I have included with the initial
 code, or with a modified version of it. Even if you modify the maven
 file (with more dependencies or plugins), this is the set of maven commands that
 need to work OK (as I will use them when checking your application):}
\begin{enumerate}
  \item {\tt mvn package}: Build jar file but first execute all the unit tests found in the test
  subfolder
   \item {\tt mvn test}: execute all the unit tests found in the test
   subfolder
  \item {\tt mvn exec:java -Dexec.args="dim tx ty numsteps stepsfilename piratesfilename"}: execute the main
  class of the program passing to the main function the required arguments.
  \item {\tt mvn javadoc:javadoc}. Generate in html files all the documentation at the level
  of classes, class functions and package general documentation.
\end{enumerate}
\item {\bf Documentation}. A document where you explain the
design of your program.
The documentation must also contain an explanation of the
propositional logical formula  that you have used to
encode the inference rules of the agent. You can write all this
documentation using javadoc comments in the different classes, and in the
documentation at the level of the application package (file {\tt package-info.java}).
Or you can provide a separate PDF file for this, using javadoc comments
only for class and class functions, as I have requested in the Requirements.
All the javadoc html documents obtained, will be found in the folder:
{\tt target/site/apidocs/apryraz/tworld/}.
\item {\bf Code}. Give all the needed code for running your
program. Use the same folder structure you have in the initial code I have provided,
or modify if you think that this is needed, but then make sure that the maven build file works
OK. {\bf Do not give a program} that needs the installation of
special  libraries that cannot be found in maven repositories (or include
the needed jar files in your code but then add the appropriate dependencies in the
maven file). So, check that you use only standard  java libraries  or that you have
 provided the needed maven dependencies, so that they will be downloaded
automatically by maven if needed.
If you use a satsolver different from sat4j, you must, of course, include it also in
your code.
\end{itemize}
\end{fullwidth}

\end{document}
